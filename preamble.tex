\usepackage[a4paper, left=30mm, right=30mm, top=20mm, bottom=20mm]{geometry}

\usepackage{graphicx}
\usepackage[export]{adjustbox}
\usepackage{epstopdf}
\usepackage{eso-pic}
%\usepackage{cite}
\usepackage{verbatim}
\usepackage{algorithm}
\usepackage{algpseudocode}
%\graphicspath{{./Graphics/}}
\usepackage{gensymb}
\usepackage{url}
\usepackage[style=nature]{biblatex}
\addbibresource{references.bib}
%\bibliography{references.bib}

\usepackage{lmodern}  % for bold teletype font
\usepackage{amsmath}  % for \hookrightarrow
\usepackage{xcolor}   % for \textcolor

\usepackage{listings}
\lstset {
	basicstyle=\footnotesize,
	columns=fullflexible,
	%frame=single,
	breaklines=true,
	postbreak=\mbox{\textcolor{red}{$\hookrightarrow$}\space},
}

\usepackage{setspace}
\linespread{1.5}

\newcommand\BackgroundPic{
\put(0,0){
\parbox[b][\paperheight]{\paperwidth}{%
\vfill
\centering
\includegraphics[width=\paperwidth,height=\paperheight,
keepaspectratio]{00_headerpic.pdf}%
\vfill
}}}

\usepackage{fancyhdr}

% Equation Formatting
\RequirePackage{suffix}
\newcommand{\neweq}[2]{
	\begin{equation}
	\label{e:#1}
	#2
	\end{equation}
}

\WithSuffix\newcommand\neweq*[1]{
	$$
	#1
	$$
}

\newcommand{\eq}[1]{(\ref{#1})}
\newcommand{\eqs}[2]{Eqs. \ref{e:#1} \& \ref{e:#2}}
\newcommand{\Eq}[1]{Equation \ref{e:#1}}

% Pretty Fractions
\RequirePackage{amsfonts}
\RequirePackage{amsmath}
\numberwithin{equation}{section}

% Images
\RequirePackage{graphicx}
\RequirePackage{wrapfig}
\RequirePackage{float}
\setkeys{Gin}{ width=\linewidth, totalheight=\textheight, keepaspectratio }

% Text wrapped figures
\newcommand{\rightfig}[3]{
	\begin{wrapfigure}{R}{.45\textwidth}
		\includegraphics[width=.4\textwidth]{#1}
		\caption{#3}
		\label{f:#2}
	\end{wrapfigure}
}
\newcommand{\leftfig}[3]{
	\begin{wrapfigure}{L}{.45\textwidth}
		\includegraphics[width=.4\textwidth]{#1}
		\caption{#3}
		\label{f:#2}
	\end{wrapfigure}
}
\newcommand{\sidebyside}[6]{
	\begin{figure}[H]
		\begin{minipage}[b]{0.45\linewidth}
			\centering
			\includegraphics[width=\textwidth]{#1}
			\caption{#3}
			\label{f:#2}
		\end{minipage}
		\hspace{0.05\linewidth}
		\begin{minipage}[b]{0.45\linewidth}
			\centering
			\includegraphics[width=\textwidth]{#4}
			\caption{#6}
			\label{f:#5}
		\end{minipage}
	\end{figure}
}
\newcommand{\centerfig}[4]{
	\begin{figure}[H]
		{\centering
			\includegraphics[width=#4\textwidth]{#1}
			\caption{#3}
			\label{#2}
		}
	\end{figure}
}
\newcommand{\centertrim}[4]{
\begin{figure}[H]
	{\centering
		\adjincludegraphics[width=#4\textwidth,trim={{.14\width} {.32\height} {.14\width} {.32\height}},clip]{#1}
		\caption{#3}
		\label{#2}
	}
\end{figure}
}

\newcommand{\centertrimnew}[4]{
	\begin{figure}[H]
		{\centering
			\adjincludegraphics[width=#4\textwidth,trim={{.15\width} {.2\height} {.15\width} {.23\height}},clip]{#1}
			\caption{#3}
			\label{#2}
		}
	\end{figure}
}

\newcommand{\centertrimnewer}[4]{
	\begin{figure}[H]
		{\centering
			\adjincludegraphics[width=#4\textwidth,trim={{0\width} {.25\height} {0\width} {.25\height}},clip]{#1}
			\caption{#3}
			\label{#2}
		}
	\end{figure}
}

\newcommand{\fig}[1]{Fig. \ref{#1}}
\newcommand{\Figure}[1]{Figure \ref{f:#1}}
\newcommand{\figs}[2]{Figs. \ref{f:#1} \& \ref{f:#2}}
